\begin{frame}{目录}
    \tableofcontents
\end{frame}

\begin{section}{值函数近似\alert{(Deep Q-Learning)}}

\begin{frame}{从表格表示到函数表示}
    \begin{figure}
        \centering
        \includegraphics[width=0.6\textwidth]{assets/Figure_chapterMap.png}
    \end{figure}
\end{frame}

\begin{frame}{从表格表示到函数表示}
    直到现在本节课涉及的所有动作价值和状态价值都是用\alert{表格}表示的。
    \begin{itemize}
        \item 状态价值:
        \begin{table}[]
            \begin{tabular}{@{}ccccc@{}}
            \toprule
            State & $s_1$& $s_2$& $\cdots$& $s_n$\\ \midrule
            Value & $v_\pi(s_1)$& $v_\pi(s_2)$& $\cdots$& $v_\pi(s_n)$\\
            \bottomrule
            \end{tabular}
        \end{table}
        \item 动作价值:
        \begin{table}[]
            \begin{tabular}{@{}cccccc@{}}
            \toprule
             & $a_1$& $a_2$& $a_3$& $a_4$& $a_5$\\ \midrule
            $s_1$ & $q_\pi(s_1,a_1)$& $q_\pi(s_1,a_2)$& $q_\pi(s_1,a_3)$& $q_\pi(s_1,a_4)$& $q_\pi(s_1,a_5)$\\
            $\vdots$ &&&$\vdots$&&  \\
            $s_9$  & $q_\pi(s_9,a_1)$& $q_\pi(s_9,a_2)$& $q_\pi(s_9,a_3)$& $q_\pi(s_9,a_4)$& $q_\pi(s_9,a_5)$\\
            \bottomrule
            \end{tabular}
        \end{table}
        \item \alert{优势}:直观,更容易理解分析
        \item \alert{缺点}:难以应对大且\alert{连续}的状态空间。
        
        主要体现在两点:1)存储空间;2)泛化能力。
    \end{itemize}
\end{frame}

\begin{frame}{从表格表示到函数表示}
    比如,我们可以用一条直线来拟合所有的状态价值:
    \begin{figure}
        \begin{tikzpicture}
            \draw[->] (0,0) -- (5,0) node[right] {$s$};
            \draw[->] (0,0) -- (0,3) node[above] {$v_\pi(s)$};
            \draw[domain=0:4,smooth,variable=\x] plot ({\x},{0.5*\x+0.5});
            
            % 补充的点
            \filldraw (1,1.2) circle (2pt) node[above left] {};
            \filldraw (2,1.3) circle (2pt) node[above left] {};
            \filldraw (3,2.3) circle (2pt) node[above left] {};
            \filldraw (4,2.2) circle (2pt) node[above left] {};

            % 在直线末端添加方程
            \node at (4.8,3) {$\hat{v}(s) = as + b$};

            % 补充坐标轴上的信息
            \draw[dashed] (1,0) -- (1,1.2) node[below] at (1,0) {$s_1$};
            \draw[dashed] (2,0) -- (2,1.3) node[below] at (2,0) {$s_2$};
            \draw[dashed] (3,0) -- (3,2.3) node[below] at (3,0) {$s_3$};
            \draw[dashed] (4,0) -- (4,2.2) node[below] at (4,0) {$s_4$};
        \end{tikzpicture}
    \end{figure}
    比如这条直线的方程是:
    \[
        \hat{v}(s,w)=as+b=[s,1]\begin{bmatrix}
            a \\
            b \\
        \end{bmatrix}=\phi^T(s)w
    \]
    其中$w$是参数向量;$\phi(s)$是特征向量;$\hat{v}(s,w)$关于$w$是线性的。
\end{frame}

\begin{frame}{从表格表示到函数表示}
    表格表示和函数表示有什么区别?
    \begin{itemize}
        \item 获取数据的方式不同:
        \begin{itemize}
            \item 表格:将行列作为索引,直接去表中查得相应数据
            \item 函数:需要输入特征向量计算:$s\rightarrow\phi(s)\rightarrow\phi^T(s)w=\hat{v}(s,w)$。
        \end{itemize}
        \item 更新数据的方式不同:
        \begin{itemize}
            \item 表格:将行列作为索引,直接去表中更新相应数据
            \item 函数:通过更新$w$来间接更新相应数据$\hat{v}(s,w)$。
        \end{itemize}
    \end{itemize}
    优点:我们不需要再存储所有的$|\mathcal{S}|$个状态价值,只需要存储$w$中的参数。
\end{frame}

\begin{frame}{从表格表示到函数表示}
    更新函数表示:
    \begin{figure}
        \centering
        \begin{tikzpicture}
            \draw[->] (0,0) -- (3,0) node[right] {$s$};
            \draw[->] (0,0) -- (0,1.5) node[above] {$v_\pi(s)$};
            
            % 补充的点
            \filldraw (0.5,0.5) circle (2pt) node[above left] {};
            \filldraw (1.5,0.5) circle (2pt) node[above left] {};
            \filldraw (2.5,0.5) circle (2pt) node[above left] {};

            % 补充坐标轴上的信息
            \draw[dashed] (0.5,0) -- (0.5,0.5) node[below] at (0.5,0) {$s_1$};
            \draw[dashed] (1.5,0) -- (1.5,0.5) node[below] at (1.5,0) {$s_2$};
            \draw[dashed] (2.5,0) -- (2.5,0.5) node[below] at (2.5,0) {$s_3$};
            
            \draw[->] (4,0.75) -- (7,0.75) node[above] at (5.5,0.75){更新$\hat{v}(s_3)$};

            \draw[->] (8,0) -- (11,0) node[right] {$s$};
            \draw[->] (8,0) -- (8,1.5) node[above] {$v_\pi(s)$};
            
            % 补充的点
            \filldraw (8.5,0.5) circle (2pt) node[above left] {};
            \filldraw (9.5,0.5) circle (2pt) node[above left] {};
            \filldraw (10.5,1) circle (2pt) node[above left] {};

            % 补充坐标轴上的信息
            \draw[dashed] (8.5,0) -- (8.5,0.5) node[below] at (8.5,0) {$s_1$};
            \draw[dashed] (9.5,0) -- (9.5,0.5) node[below] at (9.5,0) {$s_2$};
            \draw[dashed] (10.5,0) -- (10.5,1) node[below] at (10.5,0) {$s_3$};
        \end{tikzpicture}
    \end{figure}
    \begin{figure}
        \centering
        \begin{tikzpicture}
            \draw[->] (0,0) -- (3,0) node[right] {$s$};
            \draw[->] (0,0) -- (0,1.5) node[above] {$v_\pi(s)$};
            
            \draw[domain=0.25:2.75,smooth,variable=\x] plot ({\x},{0.5});

            % 补充坐标轴上的信息
            \draw[dashed] (0.5,0) -- (0.5,0.5) node[below] at (0.5,0) {$s_1$};
            \draw[dashed, color=red] (1.5,0) -- (1.5,0.5) node[below] at (1.5,0) {$s_2$};
            \draw[dashed] (2.5,0) -- (2.5,0.5) node[below] at (2.5,0) {$s_3$};
            
            \draw[->] (4,0.75) -- (7,0.75) node[above] at (5.5,0.75){更新$w$};

            \draw[->] (8,0) -- (11,0) node[right] {$s$};
            \draw[->] (8,0) -- (8,1.5) node[above] {$v_\pi(s)$};
            
            \draw[domain=8.25:10.75,smooth,variable=\x] plot ({\x},{0.25*(\x-8.5)+0.5});

            % 补充坐标轴上的信息
            \draw[dashed] (8.5,0) -- (8.5,0.5) node[below] at (8.5,0) {$s_1$};
            \draw[dashed, color=red] (9.5,0) -- (9.5,0.75) node[below] at (9.5,0) {$s_2$};
            \draw[dashed] (10.5,0) -- (10.5,1) node[below] at (10.5,0) {$s_3$};
        \end{tikzpicture}
    \end{figure}
    优点:泛化性增强。在上面的例子中,我们实际上是为了让直线更好的拟合$s_3$的状态价值去改变$w$,但是$s_2$的状态价值也跟着变化了。
\end{frame}

\begin{frame}{从表格表示到函数表示}
   获得这些优点是有代价的。代价就是状态价值无法准确的被表示出来,这也是为什么这样的方法被称为值函数\alert{近似}方法。

   我们可以通过更高维度的曲线来拟合状态价值:
   \[
        \hat{v}(s,w)=as^2+bs+c=[s^2,s,1]\begin{bmatrix}
            a \\
            b \\
            c \\
        \end{bmatrix}=\phi^T(s)w
   \]
   在这种情况下:
   \begin{itemize}
    \item 随着$w$和$\phi(s)$维度的增加,状态价值可以被拟合更精确。
    \item 虽然$\hat{v}(s,w)$对于$s$是非线性的,但是对于$w$是线性的。非线性体现在$\phi(s)$中。
   \end{itemize}
\end{frame}

\begin{frame}{从表格表示到函数表示}
    快速小结:
    \begin{itemize}
     \item 核心:使用\alert{参数化的函数}来近似状态价值和动作价值:$\hat{v}(s,w)\approx v_\pi(s)$其中$w\in \mathbb{R}^m$是参数向量
     \item 关键区别:如何获取和更新$v(s)$
     \item 优点:
    \begin{itemize}
        \item 存储:$w$的维度明显低于$|S|$
        \item 泛化性:当更新状态$s$的价值时,参数$w$会更新。所以一些其他状态的价值也会随着更新。
    \end{itemize}
    \end{itemize}
\end{frame}

\begin{frame}{目标函数}
    正式定义问题:
    \begin{itemize}
        \item $\v_\pi(s)$是状态价值的\alert{真实值};$\hat{v}(s,w)$时状态价值的\alert{估计值}。
        \item 我们的目标是找到一个最优的参数$w$,使得$\hat{v}(s,w)$尽可能的接近$\v_\pi(s)$。
    \end{itemize}
    为了找到最优的$w$,我们需要两步:
    \begin{itemize}
        \item 第一步:定义一个目标函数(损失函数)来量化$\hat{v}(s,w)$和$\v_\pi(s)$之间的差距。
        \item 第二步:通过优化算法来最小化目标函数。
    \end{itemize}
    \alert{目标函数:
        \[
            J(w)=\mathbb{E}[(v_\pi(S)-\hat{v}(S,w))^2]
        \]
    }
\end{frame}

\begin{frame}{优化算法}
    有了目标函数,下一步就是使用优化算法来优化它:
    \begin{itemize}
        \item 为了最小化目标函数$J(w)$,我们可以使用\alert{梯度下降法}。
        \[
            w_{k+1}=w_k-\alpha_k\nabla_w J(w_k)
        \]
        其中梯度:
        \[
            \begin{aligned}
                \nabla_w J(w)&=\nabla_w\mathbb{E}[(v_\pi(S)-\hat{v}(S,w))^2] \\
                &=\mathbb{E}[\nabla_w(v_\pi(S)-\hat{v}(S,w))^2]\\
                &=2\mathbb{E}[(v_\pi(S)-\hat{v}(S,w))(-\nabla_w)\hat{v}(S,w)]\\
                &=-2\mathbb{E}[(v_\pi(S)-\hat{v}(S,w))\nabla_w\hat{v}(S,w)]
            \end{aligned}
        \]
        计算梯度需要涉及计算状态的期望。状态的分布是什么?
    \end{itemize}
\end{frame}

\begin{frame}{优化算法}
    我们可以使用随机采样梯度来代替真实梯度:
    \[
        \begin{aligned}
            w_{k+1}&=w_k+\alpha_k\mathbb{E}[(v_\pi(S)-\hat{v}(S,w))\nabla_w\hat{v}(S,w)] \\
            &\downdownarrows \\
            w_{t+1}&=w_t+\alpha_t(v_\pi(s_t)-\hat{v}(s_t,w_t))\nabla_w\hat{v}(s_t,w_t)
        \end{aligned}
    \]
    其中$s_t$是$S$的一个样本。在此已经把$2\alpha_t$简化为$\alpha_t$。
    \begin{itemize}
        \item 这些采样应当是按照状态的分布采样得来的,实际上可能不是。
        \item 我们需要知道$v_\pi(s_t)$的值,实际上我们不知道。
        \item 但是我们可以通过一些估计值来代替$v_\pi(s_t)$,至少让算法先可以实现。
    \end{itemize}
\end{frame}

\begin{frame}{优化算法}
    具体而言:
    \begin{itemize}
        \item 使用值近似的Monte Carlo算法:
        
        令$g_t$为从$s_t$出发的轨迹的折扣回报。那么$g_t$就可以用来估计$v_\pi(s_t)$。
        \[
            w_{t+1}=w_t+\alpha_t(g_t-\hat{v}(s_t,w_t))\nabla_w\hat{v}(s_t,w_t)
        \]
        \item 使用值近似的TD算法:
        
        按照TD算法的思路,我们也可以用$r_{t+1}+\gamma \hat{v}{s_{t+1},w_t}$来估计$v_\pi(s_t)$
        \[
            w_{t+1}=w_t+\alpha_t(r_{t+1}+\gamma \hat{v}(s_{t+1},w_t)-\hat{v}(s_t,w_t))\nabla_w\hat{v}(s_t,w_t)
        \]
    \end{itemize}
\end{frame}

\begin{frame}{优化算法}
    \begin{block}{使用值近似的TD算法}
        \begin{algorithmic}[1]
            \State \textbf{初始化:}一个对$w$可微的函数$\hat{v}(s,w)$。初始化$w=w_0$。
            \For{每一个轮次}
                \State 依据$\pi_0(s_0)$从初始状态$s_0$开始选择动作状态$a_0$
                \If{$s_t(t=0,1,2,\cdots)$不是目标状态}
                    \State 更新$(s_t, a_t)$的动作价值:
                    \[
                        q_{t+1}(s_t,a_t)=q_t(s_t,a_t)-\alpha[q_t(s_t,a_t)-(r_{t+1}+\gamma q_t(s_{t+1},a_{t+1}))]
                    \]
                    \State \alert{更新策略$\pi_t(s_t)$}:
                    \[
                        \alert{\pi_{t+1}(s_t)=\arg\max_a q_{t+1}(s_t,a)}
                    \]
                \EndIf
            \EndFor
        \end{algorithmic}
    \end{block}
    这个算法只能用来估计某个策略下的状态价值,但是它依然是后续介绍的算法的基石。
\end{frame}

\end{section}